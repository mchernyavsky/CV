%-------------------------------------------------------------------------------
%	SECTION TITLE
%-------------------------------------------------------------------------------
\cvsection{Experience}


%-------------------------------------------------------------------------------
%	CONTENT
%-------------------------------------------------------------------------------
\begin{cventries}

%---------------------------------------------------------
  \cventry
    {Software Developer (IntelliJ Rust)} % Job title
    {JetBrains} % Organization
    {Saint Petersburg, Russia} % Location
    {Jul. 2018 - PRESENT} % Date(s)
    {
      \begin{cvitems} % Description(s) of tasks/responsibilities
        \item {Integration with external tools (build system, testing framework, code coverage tool, linters, code formatter, etc).}
        \item {Remote development support (SSH, Docker, WSL).}
        \item {Type inference (const generics, async functions, lifetimes).}
        \item {Code completion and auto-import improvements.}
        \item {Various code inspections and refactorings.}
      \end{cvitems}
    }

%---------------------------------------------------------
  \cventry
    {Research Intern (HoTT and Dependent Types group)} % Job title
    {JetBrains Research} % Organization
    {Saint Petersburg, Russia} % Location
    {Jul. 2017 - Aug. 2017} % Date(s)
    {
      \begin{cvitems} % Description(s) of tasks/responsibilities
        \item {The original creator of the plugin that brings Arend support to IntelliJ IDEA and other IntelliJ-based products. \\ Arend is a theorem prover based on Homotopy Type Theory.\href{https://github.com/JetBrains/intellij-arend}{\ \faGithub}}
      \end{cvitems}
    }

%---------------------------------------------------------
  \cventry
    {Software Developer Intern (RubyMine)} % Job title
    {JetBrains} % Organization
    {Saint Petersburg, Russia} % Location
    {Jul. 2016 - Feb. 2017} % Date(s)
    {
      \begin{cvitems} % Description(s) of tasks/responsibilities
        \item {Ruby Type Tracker - a gem to attach to Ruby processes and trace and intercept all method calls to log type-wise data flow in runtime.}
        \item {The plugin for RubyMine uses the collected type information to improve features for users such as code completion, name resolution, etc.\href{https://github.com/JetBrains/intellij-arend}{\ \faGithub}}
        \item {Sharing user-collected type information via Amazon Web Services.}
      \end{cvitems}
    }

%---------------------------------------------------------
  \cventry
    {Software Engineering Intern (Backup \& Recovery)} % Job title
    {Acronis} % Organization
    {Saint Petersburg, Russia} % Location
    {Mar. 2015 - Aug. 2015} % Date(s)
    {
      \begin{cvitems} % Description(s) of tasks/responsibilities
        \item {Developed Deduplication Datastore Index Service that stores information about unique blocks (hashes and metadata) and allows clients to know which block to deduplicate without interactions with Backup Service.}
        \item {Worked on infrastructure for testing the deduplication backup system.}
      \end{cvitems}
    }

%---------------------------------------------------------
\end{cventries}
